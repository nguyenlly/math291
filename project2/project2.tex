\documentclass[11pt]{article}
\usepackage[top=1in, bottom=1.25in, left=1.25in, right=1.25in]{geometry}

% See geometry.pdf to learn the layout options. There are lots.
%\geometry{letterpaper}                   % ... or a4paper or a5paper or ...
%\geometry{landscape}                % Activate for for rotated page geometry
\usepackage[parfill]{parskip}    % Activate to begin paragraphs with an empty line rather than an indent
%\input{preamble}       %my preamble, that I use often or go to templates
%
%PREAMBLE EG MAY 16 2005
%BEGINS
\usepackage{amsmath, amsfonts,amssymb, latexsym, mathrsfs}
\usepackage[all]{xy}
\usepackage{color}
\usepackage{graphicx}
\usepackage{graphics}
\usepackage{bigints}
\usepackage{relsize}%
\pagenumbering{gobble}

\newcommand{\ds}{\displaystyle}
\newcommand{\dx}{\; dx}
\newcommand{\comment}[1]{}

\begin{document}
\begin{center}
\noindent
{\Large \sc MATH 140 Final Exam} \\
Spring 2030\\
\vspace{20pt}

\end{center}
\noindent Name: \rule{7.5cm}{.1mm} \hfill ID:\rule{3cm}{.1mm} \vspace{1cm}\\
\vspace{2cm}\noindent Instructor: \rule{7.5cm}{.1mm}

\vspace{10pt} \textbf{Instructions:}
\begin{enumerate}
  \item This is a closed book exam.
  \item Show your answers and arguments for your answers in the space provided.
  \item Answers without proper justification \textbf{will not} receive full credit.
  \item Put your final answers in the provided answer boxes.
  \item \textbf{Calculators and electronic devices are not allowed.}
\end{enumerate}



\newpage

\pagenumbering{arabic}
%%%%%%%%%%%%%%%%%%     Question 1     %%%%%%%%%%%%%%%%%%%%

\textbf{Question 1. (24 points)} 
Evaluate the limit or show that it does not exist. If it does not exist, 
determine whether the limit is $\infty$, -$\infty$, or neither. 
   
(a)  $lim_{x\to3} \displaystyle \frac{x^2-x-6}{x^2-10x+21}$

\vspace{6.5cm}

(b)  $lim_{x\to\infty} \displaystyle \frac{2e^x+3e^{-x}}{5e^x-7e^{-x}}$

\vspace{6.5cm}

(c)  $lim_{x\to0} \displaystyle \frac{|x|}{x}$


\newpage

%%%%%%%%%%%%%%%%%%      Question 2       %%%%%%%%%%%%%%%%%%%%%
\textbf{\large Question 2. (24 points)}
Differentiate the following functions.

(a) $f(x) = \tan(x)$

\vspace{6.5cm}

(b) $f(x) = 2e^x + \frac{1}{\sqrt{x}} - x^4 + 3x^{\sqrt{5}} + 
\pi^3 + \frac{e^2}{x}$

\vspace{6.5cm}

(c) $f(x) = \big{(}\sin{(7x)} + \frac{1}{2}\cos{(x)}\big{)}^{4}$

\newpage

%%%%%%%%%%%%%%%%%%      Question 3       %%%%%%%%%%%%%%%%%%%%%%%
\textbf{\large Question 3. (24 points)}
Use the guidelines for curve sketching to sketch the graph of the
following function: 
\newline
$$
f(x) = \frac{x^3}{x-2}
$$

\vspace{6.5cm}

%%%%%%%%%%%%%%%%%%       Question 4      %%%%%%%%%%%%%%%%%%%%%%%%
\textbf{\large Question 4. (24 points)}
Let $x$ and $y$ be selected via the equation $x^2+xy+y^2=3$

(a) Using implicit differentiation, express the derivative 
$\frac{\partial y}{\partial x}$ in terms of $x$ and $y$.

\vspace{6.5cm}

(b) Find the equation of the line tangent to the curve given above 
at the point $(1,-2)$.

\newpage

%%%%%%%%%%%%%%%%%%%      Question 5      %%%%%%%%%%%%%%%%%%%%%%%%
\textbf{\large Question 5. (24 points)} 
Let $f(x) = 3x + \frac{27}{x}$. 

(a) Find the critical points of $f$.

\vspace{6.5cm}

(b) For each critical point, determine whether $f$ has a local maximum,
local minimum, or neither at that point. 

\newpage 

%%%%%%%%%%%%%%%%%%%       Question 6      %%%%%%%%%%%%%%%%%%%%%%%
\textbf{\large Question 6. (24 points)} 
Determine the maximum possible area of a rectangle of perimeter 30cm. 
(For full credit, you need to prove that your answer is the maximum
possible, using calculus.)


\newpage
%%%%%%%%%%%%%%%%%%%       Question 7      %%%%%%%%%%%%%%%%%%%%%
\textbf{\large Question 7. (24 points)}
In this problem, you must do part (a), however for full credit you
only need to \underline{complete either part (b) or part (c)}, whichever you 
choose.

(a) Consider the function $f(x) = x^3 + 5x +7$. Show that there exists 
a number $x_0$ such that $f(x_0) = 0$

\vspace{6.5cm}

(b) Find an interval of length $\frac{1}{2}$ or less containing the 
number $x_0$ from part (a). (Your answer must be explicit, like 
$7.2 \leq x \leq 7.7$, answers like $x_0 -\frac{1}{4} \leq x
\leq x_0 + \frac{1}{4}$ will not receive credit.)

\vspace{6.5cm}

(c) Prove that $x=x_0$ from part (a) is the \underline{only} real number
for which $f(x)=0$. 

\newpage

%%%%%%%%%%%%%%%%%%%      Question 8        %%%%%%%%%%%%%%%%%%%%%%%
\textbf{\large Question 8. (24 points)}
\newline
(a) Find the linear approximation of the function $f(x)= \sqrt{1-x}$ at 
the point $a=0$. 

\vspace{10cm}

(b) Use the above approximation to estimate $\sqrt{0.9}$ and $\sqrt{0.99}$.


\newpage

%%%%%%%%%%%%%%%%%%%      Question 9         %%%%%%%%%%%%%%%%%%%%%%%
\textbf{\large Question 9. (24 points)} Estimate the integral
$$\int_{-1/2}^{3/2} \frac{1}{x^4+1}dx$$
Using a Riemann sum with four equal width intervals, and by choosing the 
sample point to be the left endpoint of each interval. 
\newline

\newpage

%%%%%%%%%%%%%%%%%%%     Question 10         %%%%%%%%%%%%%%%%%%%%%%%%%
\textbf{\large Question 10. (24 points)} Find the indicated derivative 
using the Fundamental \newline Theorem of Calculus. 

(a) $\frac{\partial}{\partial x} \bigg{(} \bigint_x^1 t^2 \; dt \bigg{)}$

\vspace{4cm}
(b) $\frac{\partial}{\partial x} \bigg{(} \bigint_0^{x^2} \cos^2{(t)} \;dt
\bigg{)}$

\vspace{4cm}

%%%%%%%%%%%%%%%%%%%     Question 11     %%%%%%%%%%%%%%%%%%%%%%%%
\textbf{\large Question 11. (24 points)} Evaluate the following integrals:
\newline
(a) $\bigint x^2 \big{(} 1+{\frac{1}{\sqrt{x}} \big{)}\;dx}$

\vspace{4cm}

(b) $\bigint_{-1}^{1} x^2\sin(x^3) \;dx$

\newpage

%%%%%%%%%%%%%%%%%%%      Question 12      %%%%%%%%%%%%%%%%%%%%%%%%%%
\textbf{\large Question 12. (24 points)} Find the area of the region bounded 
by the curves $y=4x^2$ and $y=x^2+3$.


\end{document}
